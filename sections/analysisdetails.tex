\newpage
\section{Event sample, track selection and particle identification}
\label{Sec:EventTrackIdentification}
\subsection{Trigger selection and data sample}
\label{SubSec:Event}
The analysis is performed on the minimum bias Pb--Pb collision data at \sNN~collected by ALICE in 2015. These events were triggered by the coincidence between signals from both V0A and V0C detectors. An offline event selection, exploiting the signal arrival time in V0A and V0C, measured with a 1 ns resolution, was used to discriminate beam induced-background (e.g. beam gas events) from collision events. This led to a reduction of background events in the analysed samples to a negligible fraction (< 0.1\%) \cite{Abelev:2014ffa}. Events with multiple reconstructed vertices were rejected by comparing multiplicity estimates from the V0 detector to tracking detectors at mid-rapidity, exploiting the difference in readout times between the systems. The fraction of pileup events left after applying the dedicated pileup removal criteria is found to be negligible. All events selected for the analysis had a reconstructed primary vertex position along the beam axis ($z_{vtx}$) within 10 cm from the nominal interaction point. After all the selection criteria, a filtered data sample of approximately 40 million Pb--Pb events in 0-60\% centrality interval were analysed to produce the results presented in this article.

Events were classified according to fractions of the inelastic cross section. The 0-5\% interval represents the most central interactions (i.e. smallest impact parameter) and is referred to as most central collisions. On the other hand, the 50-60\% centrality interval corresponds to the most peripheral (i.e. largest impact parameter) collisions in the analysed sample. The centrality of the collision was estimated using the energy deposition measured in the V0 detectors. Details about the centrality determination can be found in Ref. \cite{Abelev:2013qoq}.

\subsection{Track selection}
\label{SubSec:Track}
In this analysis, tracks are reconstructed using the information from the TPC and the ITS detectors. The tracking algorithm, based on the Kalman filter \cite{Billoir:1983mz,Billoir:1985nq}, starts from a collection of space points (referred to as clusters) inside the TPC, and provides the quality of the fit by calculating its $\chi^{2}$ value. Each space point is reconstructed at one of the TPC padrows, where the deposited ionisation energy is also measured. The specific ionisation energy loss $\langle{\rm{dE}/dx}\rangle$ is estimated using a truncated mean, excluding the 40\% highest-charge clusters associated to the track. The obtained $\langle{\rm{dE}/dx}\rangle$ has a resolution, which we later refer to as $\sigma_{\rm{TPC}}$. The tracks are propagated to the outer layer of the ITS, and the tracking algorithm attempts to identify space points in each one of the consecutive layers, reaching the innermost ones (i.e. SPD). The track parameters are then updated using the combined information from both the TPC and the ITS detectors. 

Primary charged pions, kaons and (anti-)protons were required to have at least 70 reconstructed space points out of the maximum of 159 in the TPC. The average of the track fit per TPC space point per degree of freedom (see \cite{Abelev:2014ffa} for details) was required to be below 4. These selections reduce the contribution from short tracks, which are unlikely to originate from the primary vertex. To further reduce the contamination by secondary tracks from weak decays or from the interaction with the material, only particles within a maximum distance of closest approach (DCA) between the tracks and the primary vertex in both the transverse plane (${\rm{DCA}}_{xy} < 0.0105 + 0.0350(p_{\rm{T}}~c/{\rm{GeV}})^{-1.1}$ cm) and the longitudinal direction ($\rm{DCA}_{z} < 2$ cm) were analysed. Moreover, the tracks were required to have at least two associated ITS clusters in addition to having a hit in either of the two SPD layers. This selection leads to an efficiency of about 80\% for primary tracks at \pT$ < $ 0.6 \GeV~and a contamination from secondaries of about 5\% at \pT $= 1$ \GeV~\cite{Abelev:2013vea}. These values depend on particle species and transverse momentum \cite{Abelev:2013vea}. Relevant selection criteria for tracks used for the reconstruction of \Ks, \lambdas~and $\phi$-meson are given in Sec. \ref{SubSec:K0sLambdaRec} and \ref{SubSec:PhiRec}.

\subsection{Identification of \pion, \kaon~$\rm{and}$ \proton}
\label{SubSec:Identification}
%\pion, \kaon~$\rm{and}$ \proton~identification
%(\pion), kaons (\kaon) and protons (\proton)
The particle identification (PID) for pions (\pion), kaons (\kaon) and protons (\proton) used in this analysis relies on the two-dimensional correlation between the number of standard deviations in units of the resolution from the expected signals of the TPC and the TOF detectors similar to what was reported in \cite{Abelev:2014pua,Adam:2016nfo,Acharya:2018zuq}. In this approach particles were selected by requiring their standard deviations from the $\langle d{\it E}/d{\it x}\rangle$ and $t_{\rm TOF}$ values to not only be the minimum among other candidate species and lie within a $p_{\rm T}$-dependent maximum combined standard deviations for a given particle species and transverse momentum maintaining a minimum purity of 90\% for \pion, 75\% for \kaon~and 80\% for \proton. 

%their signal to lie within a $p_{\rm T}$-dependent maximum standard deviations from the $\langle d{\it E}/d{\it x}\rangle$ and $t_{\rm TOF}$ values expected for a given particle species and transverse momentum maintaining a minimum purity of 90\% for \pion, 75\% for \kaon~and 80\% for \proton. 

In addition, for the systematics (see section \ref{Sec:Systematics}) the minimum purity was required to be 80\%, a condition that becomes essential with increasing transverse momentum where the relevant detector response for different particle species starts to overlap.

\subsection{Reconstruction of \Ks~and \lambdas}
\label{SubSec:K0sLambdaRec}

%The \Ks~and \lambdas~are electrically neutral particles which undergo a weak decay into a pair of the daughter products with opposite charges: \Ks $\rightarrow \pi^{+} + \pi^{-}$ or \lambdas $\rightarrow p(\bar{p})+\pi^{-}(\pi^{+})$ with branching ratios of 69.2\% \cite{Olive_2016} and 63.9\% \cite{Olive_2016}, respectively. These particles are reconstructed by identifying secondary vertices from which two oppositely-charged particles originate, namely \vo s. These candidates are obtained during the data processing based on the tracks geometry. 

%This pre-selected sample contains both true \vo~candidates and combinatorial background. The combinatorial background is suppressed by applying further criteria on characteristic "V"-shaped decay topology and reconstruction requirements on both daughters to eliminate \vo~candidates with low tracking quality.

In this analysis, the \Ks~and \lambdas~are reconstructed via the following fully hadronic decay channels: \Ks $\rightarrow \pi^{+} + \pi^{-}$ and  $\Lambda(\bar{\Lambda})\rightarrow p(\bar{p})+\pi^{-}(\pi^{+})$ with branching ratios of 69.2\% and 63.9\% \cite{Olive_2016}, respectively. The reconstruction is performed by identifying the candidates of secondary vertices denoted as \vo s, from which two oppositely-charged decay products originate. Such candidates are obtained during data processing by looking for characteristic V-shaped decay topology among reconstructed tracks.

The daughter tracks were reconstructed within $|\eta|<0.8$, while the criteria on the number of TPC space points, the number of crossed TPC padrows, and the percentage of the expected TPC space points used to reconstruct a track are identical to those applied for primary particles. In addition, the minimum DCA of daughter tracks to the primary vertex is 0.1 cm. Furthermore, the maximum DCA of daughter tracks to the secondary vertex is 0.5 cm to ensure that they are products of the same decay. To suppress the combinatorial background, the PID is applied for the daughter particles in the whole \pT~region by requiring the particle to be within 3$\sigma_{\rm TPC}$ for a given species hypothesis.

To reject secondary vertices arising from decays into more than two particles, the cosine of the pointing angle, $\theta_{p}$, is required to be larger than 0.998. This angle is defined as the angle between the momentum vector of the \vo~candidate assessed at its decay vertex and the line connecting the \vo~decay vertex to the primary vertex and has to be close to 1 as a result of momentum conservation. In addition, only the candidates reconstructed between 5 and 100 cm from the nominal primary vertex in radial direction are accepted. The lower value is chosen to avoid any bias from the efficiency loss when secondary tracks are being wrongly matched to clusters in the first layer of the ITS. To assess the systematic uncertainty related to the contamination from \lambdas~and electron-positron pairs coming from $\gamma$-conversions to the \Ks~sample, a selection in the Armenteros-Podolanski variables \cite{doi:10.1080/14786440108520416} is applied for the \Ks~candidates, rejecting ones with $q\le 0.2|\alpha|$. Here q is the momentum projection of the positively charged daughter track in the plane perpendicular to the \vo~momentum and $\alpha = (p_{\rm{L}}^{+} - p_{\rm{L}}^{-})/(p_{\rm{L}}^{+} + p_{\rm{L}}^{-})$ with $p_{\rm{L}}^{\pm}$ the projection of the positive or negative daughter tracks' momentum onto the momentum of the
\vo.

To obtain the \pT-differential yield of \Ks~and \lambdas~(which, together with background yields), are used for the signal extraction in Eq. \ref{Eq:dnmk}, invariant mass distributions at various \pT~intervals are parametrized as a sum of two Gaussian distribution and a third-order polynomial function. The latter is introduced to account for residual contamination (background yield) that are present in the \Ks~and \lambdas~signals after the topological and daughter track selections. The \Ks~and \lambdas~yields are extracted by integration of the Gaussian distribution. %The v_{n,mk}(\pT) results are reported for 1 < \pT < 6 GeV/c for \Ks and 1 < \pT < 6 GeV/c for \lambdas.

\subsection{Reconstruction of $\phi$ meson}
\label{SubSec:PhiRec}

The reconstruction of $\phi$ meson candidates is done via its following hadronic decay channel: $\varphi \rightarrow {\rm K}^{+} + {\rm K}^{-}$ with a branching ratio of 48.9\% \cite{Olive_2016}. The $\phi$ meson candidates were reconstructed from the charged tracks passing all criteria for charged kaons, listed in Table~\ref{SysVar}. Kaon daughters are identified by utilising Bayesian PID approach with a minimum probability threshold of $85\%$ using TPC and TOF detectors. Additionally, to reduce combinatorial background of $\phi$ candidates, a track is identified as kaon if it has the highest probability among all considered species ($e$,$\mu$,\pion,\kaon,\proton). The vector sum of all possible pairs of charged kaons are called $\phi$~candidates. The invariant mass distribution (${\rm M}_{inv}^{\rm K^{+}K^{-}}$) of $\phi$~candidates is then obtained in various \pT~intervals by subtracting a combinatorial background yield from the candidate yield. This combinatorial background yield is estimated from like-sign kaon pairs (unphysical $\phi$ state with total charge of $\pm2$) normalised to the candidate yield. To obtain the \pT-differential yield of $\phi$-mesons, the invariant mass distributions of the candidate yield is, after the aforementioned subtraction, parametrized as a sum of a Breit-Wigner distribution and a third-order polynomial function, the latter introduced to account for residual contamination.



%one for unlike-sign and the other for like-sign pairs. The like-sign kaon pairs represents purely the combinatorial background (unphysical $\phi$ state with total charge of $\pm2$). The unlike-sign pairs consist of two components: the decay of true $\phi$ mesons (representing the signal); and additional combinatorial background as well.

