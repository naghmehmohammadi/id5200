\newpage

\section{Summary}
\label{Sec:conclusion}

In this article, the measurement of non-linear flow modes, $v_{4,22}$, $v_{5,32}$, $v_{6,222}$ and $v_{6,33}$ as a function of transverse momentum for different particle species, i.e. \pion, \kaon, \Ks,\proton, \lambdas~and $\phi$-meson are reported for the first time. The results are presented in a wide range of centrality intervals from 0-5\% up to 50-60\% in Pb--Pb collisions at \sNN. The magnitude of non-linear flow modes, $v_{n,mk}$, are obtained with a multi-particle correlation technique, namely the generic framework, selecting the identified hadron under study and the reference flow particles from different, non-overlapping pseudorapidity regions.  

The measurements of $v_{4,22}$, $v_{5,32}$ and $v_{6,222}$ exhibit a clear centrality dependence. This centrality dependence originates from the contribution of initial state eccentricity, $\varepsilon_{2}$, as shown in Eq. \ref{Eq:V4V5V6}. As expected, $v_{6,33}$ does not exhibit a considerable centrality dependence since $\varepsilon_{3}$ quantifies primarily the event-by-event fluctuations of the initial energy density profile. This is supported by the relatively large magnitude of $v_{6,33}$ in the most-central collisions (0-5\%). A clear mass ordering is observed in the low \pT~region (\pT$< 2.5$ \GeV). This mass ordering is similar to observations in $v_{n}$ and it is associated with the interplay between the anisotropic flow and radial flow. In the intermediate \pT~region (\pT$> 2.5$ \GeV), a particle type grouping is observed where the magnitude of non-linear modes for baryons are larger than for mesons similar to observations in $v_{n}$ measurements. The NCQ scaling holds at an approximate level of $\pm 20$\% within the current level of statistical and systematic uncertainties, similar to that of anisotropic flow coefficients \cite{Acharya:2018zuq}. 

The comparison of two models based on the iEBE-VISHNU hybrid model, and with two different initial conditions (AMPT and TRENTo) and transport properties show that neither of the models are able to fully describe the measurements. This varies depending on the centrality percentile and particle species similar to the model-data comparison for anisotropic flow \cite{Acharya:2018zuq}. Measurements are better predicted by the models in more central collisions. All in all, the model using AMPT initial conditions ($\eta/s = 0.08$ and $\zeta/s =0$) exhibits a magnitude and shape closer to the measurements. As a result, in order to further constrain the values of transport properties and the initial conditions of the system, it is necessary to tune the input parameters of future hydrodynamic calculations attempting to describe these measurements.