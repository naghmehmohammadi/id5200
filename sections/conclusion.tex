\newpage

\section{Summary}
\label{Sec:conclusion}

In this article, a measurement of non-linear flow modes, $v_{4,22}$, $v_{5,32}$, $v_{6,222}$ and $v_{6,33}$ as a function of transverse momentum for different particle species, i.e. \pion, \kaon, \proton, \Ks, \lambdas~and $\phi$-meson are reported for a wide range of centrality intervals from 0-5\% up to 50-60\% in Pb--Pb collisions at \sNN. The non-linear flow modes, $v_{n,mk}$, are calculated with a multi-particle correlation technique, namely the generic framework, selecting the identified hadron under study and the reference flow particles from different, non-overlapping pseudorapidity regions. This multi-particle correlation technique by nature removes majority of non-flow correlations. In order to reduce non-flow contributions further, a non-zero gap was applied between the two pseudorapidity regions as well as selecting like sign particles of interest and reference particles. These variations did not affect the results significantly but any variation was included in the final systematics. 

The magnitude of $v_{4,22}$, $v_{5,32}$ and $v_{6,222}$ exhibit a clear centrality dependence. This centrality dependence originates from the contribution of second order flow harmonic, as shown in Eq. \ref{Eq:V4V5V6}, and reflects the dependence of $v_{2}$ on the anisotropy of the collision geometry. As expected, $v_{6,33}$ does not exhibit a considerable centrality dependence since $v_{3}$ is primarily generated by event-by-event fluctuations of the initial energy density profile. This is supported by the relatively large magnitude of $v_{6,33}$ in the most-central collisions (0-5\%). A clear mass ordering is observed in the low \pT~region (\pT$< 2.5$ \GeV). This observation is associated with the interplay between the anisotropic flow and radial flow. In the intermediate \pT~region (\pT$> 2.5$ \GeV), a particle type grouping is observed where the magnitude of non-linear modes for baryons are larger than for mesons. The NCQ scaling holds at best in an approximate level of $\pm 20$\% within the current level of statistical and systematic uncertainties similar to that of total flow coefficients \cite{Acharya:2018zuq}.

The comparison of two models based on the iEBE-VISHNU hybrid model, and with two different initial conditions (AMPT and TRENTo) and transport properties show that neither of the models are able to fully describe the measurements. This varies depending on the centrality percentile. Measurements are better predicted by the models in more central collisions. All in all, the model using AMPT initial conditions ($\eta/s = 0.08$ and $\zeta/s =0$) exhibits a magnitude and shape closer to the measurements. As a result, in order to further constrain the values of transport properties and the initial conditions of the system, it is necessary to tune the input parameters of future hydrodynamic calculations attempting to describe these measurements.