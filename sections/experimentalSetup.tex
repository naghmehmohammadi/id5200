\section{Experimental setup}
\label{Sec:ExpSetup}
ALICE~\cite{Aamodt:2008zz,Abelev:2014ffa} is one of the four large experiments at the LHC, particularly designed to cope with the large charged-particle densities present in central Pb--Pb collisions~\cite{Aamodt:2010pb}. By convention, the $z$-axis is parallel to the beam direction, the $x$-axis is horizontal and points towards the centre of the LHC, and the $y$-axis is vertical and points upwards. The apparatus consists of a set of detectors located in the central barrel, positioned inside a solenoidal magnet which generates a maximum of $0.5$~T field parallel to the beam direction, and a set of forward detectors. 

The Inner Tracking System (\ITS)~\cite{Aamodt:2008zz} and the Time Projection Chamber (\TPC)~\cite{Alme:2010ke} are the main tracking detectors of the central barrel. The \ITS~consists of six layers of silicon detectors employing three different technologies. The two innermost layers, positioned at $r = 3.9$~cm and 7.6~cm,  are Silicon Pixel Detectors (\SPD), followed by two layers of Silicon Drift Detectors (\SDD) ($r = 15$~cm and 23.9~cm). Finally, the two outermost layers are double-sided Silicon Strip Detectors (\SSD) at $r = 38$~cm and 43~cm. The \TPC~has a cylindrical shape with an inner radius of about 85 cm, an outer radius of about 250 cm, and a length of 500 cm and it is positioned around the \ITS. It provides full azimuthal coverage in the pseudorapidity range $|\eta| < 0.9$. 

Charged particles were identified using the information from the \TPC~and the \TOF~detectors~\cite{Aamodt:2008zz}. The \TPC~allows for a simultaneous measurement of the momentum of a particle and its specific energy loss ($\langle \mathrm{d}E/\mathrm{d}x \rangle$) in the gas. The detector provides a separation more than two standard deviations ($2\sigma$) for different hadron species at $p_{\mathrm{T}} < 0.7$~GeV/$c$ and the possibility to identify particles on a statistical basis in the relativistic rise region of $\mathrm{d}E/\mathrm{d}x$ (i.e.~$2 < p_{\rm{T}} < 20$~GeV/$c$)~\cite{Abelev:2014ffa}. The $\mathrm{d}E/\mathrm{d}x$ resolution for the 5$\%$ most central Pb--Pb collisions is 6.5$\%$ and improves for more peripheral collisions~\cite{Abelev:2014ffa}. The \TOF~detector is situated at a radial distance of 3.7 m from the beam axis, around the \TPC~and provides a $3\sigma$ separation between $\pi$--K and K--$\rm{p}$ up to $p_{\mathrm{T}} = $ 2.5~GeV/$c$ and $p_{\mathrm{T}} = 4$~GeV/$c$, respectively~\cite{Abelev:2014ffa}. This is done by measuring the flight time of particles from the collision point with a resolution of about $80$~ps. The start time for the \TOF~measurement is provided by the T0 detectors, two arrays of Cherenkov counters positioned at opposite sides of the interaction points covering $4.6 < \eta < 4.9$ (T0A) and $-3.3 < \eta < -3.0$ (T0C). The start time is also determined using a combinatorial algorithm that compares the timestamps of particle hits measured by the TOF to the expected times of the tracks, assuming a common event time $t_{ev}$. Both methods of estimating the start time are fully efficient for the 80\% most central Pb--Pb collisions \cite{Abelev:2014ffa}.

A set of forward detectors, the \VZERO~scintillator arrays~\cite{Abbas:2013taa}, were used in the trigger logic and for the determination of the collision centrality. The \VZERO~consists of two detectors, the \VZEROA~and the \VZEROC, positioned on each side of the interaction point, covering the pseudorapidity intervals of $2.8 < \eta < 5.1$ and $-3.7 < \eta < -1.7$, respectively. 

For more details on the ALICE apparatus and the performance of the detectors, see Refs.~\cite{Aamodt:2008zz,Abelev:2014ffa}.
