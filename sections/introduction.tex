\section{Introduction}
\label{Sec:Introduction}

Lattice quantum chromodynamics (QCD) calculations \cite{Borsanyi:2010cj,Bhattacharya:2014ara} suggest that at extremely high temperature and energy density a state of matter is produced in which quarks and gluons are no longer confined into hadrons. This state of matter is called the quark-gluon plasma (QGP) \cite{Shuryak:1984nq, Cleymans:1985wb, Bass:1998vz}. The main goal of heavy-ion collision experiments is to study the properties of the QGP, such as the speed of the sound, the equation of state and its shear and bulk viscosities.

One of the observables sensitive to these properties is the azimuthal angular distribution of particles emitted in the plane perpendicular to the beam axis. In a heavy ion collision, the overlap region of the colliding nuclei exhibits an irregular shape \cite{Miller:2003kd,Bhalerao:2006tp, Alver:2008zza, Alver:2010gr, Alver:2010dn, Manly:2005zy, Voloshin:2006gz}. This spatial irregularity is a superposition of the geometry, i.e. centrality of the collision reflected in the value of the impact parameter, and the initial energy density in the transverse plane which fluctuates from event to event. Through interactions between partons and at later stages between the produced particles, this spatial irregularity is transferred into an anisotropy in momentum space. The latter is usually expressed by a Fourier expansion of the azimuthal particle distribution \cite{Voloshin:1994mz} according to

\begin{equation}
\frac{\mathrm{d}N}{d\varphi} \propto 1+2\sum_{n=1}^{\infty} v_n(p_{\mathrm{T}}) \cos[n(\varphi - \Psi_n)],
\label{Eq:Fourier}
\end{equation}


%\begin{equation}
%E\frac{\mathrm{d}^3N}{\mathrm{d}p^3} = \frac{1}{2\pi}\frac{\mathrm{d}^2N}{p_{\mathrm{T}}\mathrm{d}p_{\mathrm{T}}\mathrm{d}\eta} \Big\{1 + 2\sum_{n=1}^{\infty} v_n(p_{\mathrm{T}},\eta) \cos[n(\varphi - \Psi_n)]\Big\},
%\label{Eq:Fourier}
%\end{equation}

\noindent where $N$, $p_{\mathrm{T}}$ and $\varphi$ are the particle yield, transverse momentum and azimuthal angle of particles, respectively, and $\Psi_n$ is the azimuthal angle of the $n^{\mathrm{th}}$-order symmetry plane~\cite{Voloshin:2006gz,Bhalerao:2006tp,Alver:2008zza,Alver:2010gr,Alver:2010dn}. The coefficient $v_{n}$ is the magnitude of the $n^{\mathrm{th}}$-order complex flow vector $V_n$, defined as $V_{n} = v_{n}e^{in\Psi_n}$, and can be calculated according to 

%\noindent where $E$, $N$, $p$, $p_{\mathrm{T}}$, $\varphi$ and $\eta$ are the energy, particle yield, total momentum, transverse momentum, azimuthal angle and pseudorapidity of particles, respectively, and $\Psi_n$ is the azimuthal angle of the symmetry plane of the $n^{\mathrm{th}}$-order coefficient~\cite{Bhalerao:2006tp,Alver:2008zza,Alver:2010gr,Alver:2010dn}. The parameter $v_{n}$ is the magnitude of the $n^{\mathrm{th}}$-order complex flow coefficient $V_n$, defined as $V_{n} = v_{n}e^{in\Psi_n}$, and can be calculated according to 

\begin{equation}
v_{n} = \langle{\cos[n(\varphi - \Psi_n)]}\rangle,
\label{Eq:vn}
\end{equation}

where the brackets denote an average over all particles in all events. Since the symmetry planes are not accessible experimentally, the flow coefficients are estimated solely from the azimuthal angles of the particles emitted in the transverse plane. Measurements of different anisotropic flow coefficients at both RHIC \cite{Adams:2003am,Abelev:2007qg,Adler:2003kt,Adare:2006ti,Alver:2007qw,Adcox:2002ms,Adamczyk:2013gw,Adler:2004cj,Afanasiev:2009wq,Adare:2011tg,Ackermann:2000tr,Adler:2001nb,Adler:2002ct,Adler:2002pu,Adams:2003zg,Adams:2004wz,Adams:2004bi} and the LHC \cite{Aamodt:2010pa, ALICE:2011ab, Abelev:2012di, Chatrchyan:2012xq, Chatrchyan:2012ta, ATLAS:2011ah, ATLAS:2012at,Abelev:2014pua,Adam:2016nfo,Adam:2016izf, Acharya:2018zuq,
Chatrchyan:2013kba, Adam:2015eta, Acharya:2018lmh,Acharya:2018ihu} have not only confirmed the production of a strongly coupled quark gluon plasma (sQGP) but have also contributed in constraining the value of its shear viscosity over entropy density ($\eta/s$) which is very close to the lower limit of $1/4\pi$ conjectured by AdS/CFT \cite{Kovtun:2004de}. In addition, the comparison between experimental data \cite{Adam:2016izf} and viscous hydrodynamical calculations \cite{Niemi:2015voa} showed that higher order flow coefficients and more importantly their transverse momentum dependence are more sensitive probes than lower order coefficients, i.e. \vtwo~and \vthree, to the initial spatial irregularity and its fluctuations \cite{Alver:2010dn}.\\ % since higher order flow coefficients, originating from fluctuations of the initial state, probe smaller spatial scales.
%The non-vanishing higher order flow coefficients, $v_n (n>2)$, are believed to originate mainly from the fluctuations in the initial energy density profile of the colliding nucleons \cite{}. Thus, higher order flow coefficients are better probes to to understand the initial density profile of the 
This initial state spatial irregularity is usually quantified with the standard (moment-defined) anisotropy coefficients, $\epsilon_{n}$. In the Monte-Carlo Glauber model, $\epsilon_{n}$ and its corresponding initial symmetry plane, $\Phi_n$ can be calculated from the transverse positions of the nucleons participating in a collision according to \cite{Teaney:2010vd, Alver:2010gr}

\begin{equation}
\epsilon_{n}e^{in\Phi_n} = \frac{\langle{r^{n}e^{in\phi}}\rangle}{\langle r^n\rangle}  (\rm{for}~n>1),
\label{Eq:epsilonn}
\end{equation}

where the brackets denote an average over the transverse position of all participating nucleons that have an azimuthal angle $\phi$ and a polar distance from the centre $r$. Model calculations show that $v_2$ and to a large extent, $v_3$ are for a wide range of impact parameters linearly proportional to their corresponding initial spatial anisotropy coefficients, $\epsilon_{2}$ and $\epsilon_{3}$, respectively \cite{Alver:2010gr} while for larger values of $n$, $v_{n}$ scales with a cumulant-based definition of initial anisotropic coefficients. %This definition suggests additional terms for $\epsilon_{n}$ in higher order flow coefficients ($n>3$). 
As an example, the fourth order spatial anisotropy is given by 

%nonlinearities are observed , i.e. $v_{n} \not\propto \epsilon_{n}$ \cite{Alver:2010dn}.
%Alver:2010gr,Alver:2010dn} 
%A cumulant-based definition of initial anisotropic coefficients suggests additional terms in the definition of $\epsilon_{n}$ for higher order flow coefficients ($n>3$). As an example, the fourth order spatial anisotropy is given by 
 
\begin{equation}
\epsilon_{4}'e^{i4\Phi'_4} = \epsilon_{4}e^{i4\Phi_4}  + \frac{3\langle{r^{2}}\rangle^{2}}{\langle r^4\rangle}\epsilon_{2}^{2}e^{i4\Phi_2},
\label{Eq:epsilonnprime}
\end{equation}

where $\epsilon_{4}'$ is the cumulant-based spatial anisotropy coefficient \cite{Teaney:2013dta,Qian:2017ier}, and the second term in the right hand side of Eq. \ref{Eq:epsilonnprime} reveals a non-linear dependence of $\epsilon_{4}'$ on the lower order $\epsilon_{2}$. This further supports the earlier ideas that the higher order flow coefficients, $V_n~(n > 3)$ obtain contributions not only from the linear response of the system to $\epsilon_{n}$, but also a non-linear response proportional to the product of lower order initial spatial anisotropies \cite{Bhalerao:2014xra,Yan:2015jma}. 

In particular, for a single event, $V_n$ with $n=4,5,6$ can be decomposed to the linear ($V_{n}^{\mathrm{L}} $) and non-linear ($ V_{n}^{\mathrm{NL}}$) modes according to%in \cite{Acharya:2017zfg}

\vspace{-0.55cm}
\begin{align}
V_{4} &= V_{4}^{\mathrm{L}} + V_{4}^{\mathrm{NL}} = V_{4}^{\mathrm{L}} + \chi_{4,22}(V_{2})^2, \nonumber \\
V_{5} &= V_{5}^{\mathrm{L}} + V_{5}^{\mathrm{NL}} = V_{5}^{\mathrm{L}} + \chi_{5,32}V_{3}V_{2}, \nonumber \\
V_{6} &= V_{6}^{\mathrm{L}} + V_{6}^{\mathrm{NL}} = V_{6}^{\mathrm{L}} + \chi_{6,222}(V_{2})^3 + \chi_{6,33}(V_{3})^2 + \chi_{6,42}V_{2}V_{4}^{\mathrm{L}},
\label{Eq:V4V5V6}
\end{align}
\vspace{-0.55cm}

where $\chi_{n,mk}$, known as non-linear flow mode coefficients, quantify the contributions of the non-linear modes to the total $V_{n}$ \cite{Yan:2015jma, Acharya:2017zfg}. For simplicity the magnitude of the total $V_{n}$ will be referred to as anisotropic flow coefficient ($v_{n}$) in the rest of this article. 
The magnitude of the $p_{\rm{T}}$-differential non-linear modes for higher order flow coefficients, $v_{n}^{\rm{NL}}$, can be written as: 

\begin{align}
v_{4,22}(p_{\rm{T}})&= \frac{\langle v_{4}(p_{\rm{T}})v_{2}^{2}\cos(4\Psi_{4}-4\Psi_{2})\rangle}{\sqrt{\langle v_{2}^{4}\rangle}} \approx \langle v_{4}(p_{\rm{T}})\cos(4\Psi_{4}-4\Psi_{2})\rangle, \label{Eq:V422}\\
v_{5,32}(p_{\rm{T}})&= \frac{\langle v_{5}(p_{\rm{T}})v_{3}v_{2}\cos(5\Psi_{5}-3\Psi_{3}-2\Psi_{2})\rangle}{\sqrt{\langle v_{3}^{2}v_{2}^{2}\rangle}} \approx \langle v_{5}(p_{\rm{T}})\cos(5\Psi_{5}-3\Psi_{3}-2\Psi_{2})\rangle, \label{Eq:V532}\\
v_{6,33}(p_{\rm{T}}) &= \frac{\langle v_{6}(p_{\rm{T}})v_{3}^{2}\cos(6\Psi_{6}-6\Psi_{3})\rangle}{\sqrt{\langle v_{3}^{4}\rangle}} \approx \langle v_{6}(p_{\rm{T}})\cos(6\Psi_{6}-6\Psi_{3})\rangle , \label{Eq:V633}\\
v_{6,222}(p_{\rm{T}}) &= \frac{\langle v_{6}(p_{\rm{T}})v_{2}^{3}\cos(6\Psi_{6}-6\Psi_{2})\rangle}{\sqrt{\langle v_{2}^{6}\rangle}} \approx \langle v_{6}(p_{\rm{T}})\cos(6\Psi_{6}-6\Psi_{2})\rangle,
\label{Eq:V6222}
\end{align}

\noindent where brackets denote an average over all events. The approximation is valid assuming a weak correlation between the lower ($n=2,3$) and higher ($n>3$) order flow coefficients \cite{Acharya:2017gsw, Bhalerao:2014xra}.

Various measurements of the \pT-differential anisotropic flow, $v_{n}$(\pT), of charged particles \cite{Voloshin:2008dg, ALICE:2011ab, ATLAS:2012at, Chatrchyan:2013kba, Acharya:2018lmh,Acharya:2018ihu} have provided a testing ground for model calculations that attempt to describe the dynamical evolution of the system created in heavy-ion collisions. Early predictions showed that the \pT-differential anisotropic flow for different particle species can reveal more information about the equation of state, the role of the highly dissipative hadronic rescattering phase as well as probing particle production mechanisms \cite{Voloshin:1996nv,Huovinen:2001cy}. In order to test these predictions, $v_{n}$(\pT) have been measured for different particle species at RHIC \cite{Adams:2003am,Abelev:2007qg,Adler:2003kt,Adare:2006ti} and at the LHC \cite{Abelev:2014pua,Adam:2015eta,Adam:2016nfo,Acharya:2018zuq}. These measurements reveal a characteristic mass dependence of $v_{n}$(\pT) in the low transverse momentum region ($p_{\rm{T}} < 3~{\rm GeV}/c$), a result of an interplay between radial and anisotropic flow, and mass dependent thermal velocities \cite{Voloshin:1996nv,Huovinen:2001cy}. 
%These measurements have revealed that an interplay between radial and anisotropic flow leads to a characteristic mass dependence in the low transverse momentum region ($p_{\rm{T}} < 3~{\rm GeV}/c$). 
In the intermediate \pT~region (\pT~$\sim$~3 \GeV) the measurements indicate a particle type grouping where baryons have a larger $v_{n}$ than the one of mesons. This feature was explained in a dynamical model where flow develops at the partonic level followed by quark coalescence into hadrons \cite{Voloshin:2002wa,Molnar:2003ff}. In this picture the invariant spectrum of produced particles is proportional to the product of the spectra of their constituents and, in turn, the flow coefficient of produced particles is the sum of the $v_{n}$ values of their constituents. This leads to the so-called number of constituent quarks (NCQ) scaling, observed to hold at an approximate level of $\pm20$\% for $p_{\rm{T}} > 3$ \GeV~\cite{Adare:2006ti,Adare:2012vq,Abelev:2014pua,Adam:2016nfo}.
%Measurements of lower order anisotropic flow coefficients exhibit what is usually referred to as number of constituent quarks (NCQ) scaling at RHIC \cite{Adare:2012vq} and the LHC \cite{Abelev:2014pua,Adam:2016nfo} at an approximate level of $\pm20$\% for $p_{\rm{T}} > 3$ \GeV.

The measurements of non-linear flow modes in different collision centralities could pose a challenge to hydrodynamic models and have the potential to further constrain both the initial conditions of the collision system and its transport properties, i.e. $\eta/s$ and $\zeta/s$ \cite{Zhu:2016puf, Acharya:2017zfg}. The \pT-dependent non-linear flow modes of identified particles, in particular, allow to test the effect of late-stage interactions in the hadronic rescattering phase, as well as the effect of particle production via the coalescence mechanism to the development of the mass ordering at low \pT~and particle type grouping in the intermediate \pT~region, respectively \cite{ALICE:2011ab,Acharya:2018zuq}.


%The \pT-dependent non-linear flow modes of identified particles put a stringent constraint on the initial conditions of the collision system and its transport properties. In addition, they allow to test the effect of late-stage interactions in the hadronic rescattering phase, as well as the effect of particle production via the coalescence mechanism to the development of the mass ordering and particle type grouping, respectively \cite{ALICE:2011ab,Acharya:2018zuq}.

In this article, we report the first results of the $p_{\rm{T}}$-differential non-linear flow modes, i.e. $v_{4,22}$, $v_{5,32}$, $v_{6,33}$ and $v_{6,222}$ for \pion, \kaon, \Ks, \proton, \lambdas~and $\varphi$ measured in Pb--Pb collisions at a centre of mass energy per nucleon pair \sNN, recorded by the ALICE experiment \cite{Aamodt:2008zz} at the LHC. The detectors and the selection criteria used in  this analysis are described in Sec. \ref{Sec:ExpSetup} and \ref{Sec:EventTrackIdentification}, respectively. %The charged particles are identified using signals from both the Time Projection Chamber (TPC) and the Time Of Flight (TOF) detectors described in Section \ref{Sec:ExpSetup} with the procedures in Section \ref{SubSec:Identification} and neutral particles on a statistical basis using an invariant mass technique, also described in Section \ref{SubSec:Identification}. 
The analysis methodology and technique are presented in Sec. \ref{Sec:Analysis method}. In this article, the identified hadron under study and the charged reference particles are obtained from different, non-overlapping pseudorapidity regions. The azimuthal correlations not related to the common symmetry plane (known as non-flow), including the effects arising from jets, resonance decays and quantum statistics correlations, are suppressed by using multi-particle correlations as explained in Sec. \ref{Sec:Analysis method} and the residual effect has been taken into account in the systematic uncertainty, described in Sec. \ref{Sec:Systematics}. All coefficients for charged particles were measured separately for particles and anti-particles and were found to be compatible within statistical uncertainties. The measurements reported in Sec. \ref{Sec:Results} are therefore an average of the results for both charges. The results are reported within the pseudorapidity range $|\eta|<0.8$ at different collision centralities between 0--60\% range of Pb--Pb collisions. 







