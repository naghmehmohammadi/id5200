\section{Introduction}
\label{Sec:Introduction}

Lattice quantum chromodynamics (QCD) calculations \cite{Borsanyi:2010cj,Bhattacharya:2014ara} suggest that at extremely high temperature and energy density a state of matter is produced in which quarks and gluons are no longer confined to hadrons. This state of matter is called quark-gluon plasma (QGP) \cite{Shuryak:1984nq, Cleymans:1985wb, Bass:1998vz}. The main goal of heavy-ion collision experiments is to study the properties of the QGP, such as the speed of the sound in this medium, the equation of state and its transport properties, e.g. shear and bulk viscosity.

One of the observables sensitive to the properties of the QGP is the azimuthal angular distribution of particles emitted in the plane perpendicular to the beam axis. In a heavy ion collision, the overlap region of the colliding nuclei exhibits an irregular shape \cite{Bhalerao:2006tp, Alver:2008zza, Alver:2010gr, Alver:2010dn, Manly:2005zy}. This spatial irregularity is a superposition of the geometry, i.e. centrality of the collision, and the anisotropic initial density profile of nucleons participating in the collision. Through interactions between partons and at later stages between the produced particles, this spatial irregularity is transferred into the anisotropy in momentum space which is usually expressed by the Fourier expansion of the azimuthal particle distribution according to

\begin{equation}
E\frac{\mathrm{d}^3N}{\mathrm{d}p^3} = \frac{1}{2\pi}\frac{\mathrm{d}^2N}{p_{\mathrm{T}}\mathrm{d}p_{\mathrm{T}}\mathrm{d}\eta} \Big\{1 + 2\sum_{n=1}^{\infty} v_n(p_{\mathrm{T}},\eta) \cos[n(\varphi - \Psi_n)]\Big\},
\label{Eq:Fourier}
\end{equation}

\noindent where $E$, $N$, $p$, $p_{\mathrm{T}}$, $\varphi$ and $\eta$ are the energy, particle yield, total momentum, transverse momentum, azimuthal angle and pseudorapidity of particles, respectively, and $\Psi_n$ is the azimuthal angle of the symmetry plane of the $n^{\mathrm{th}}$-order coefficient~\cite{Bhalerao:2006tp,Alver:2008zza,Alver:2010gr,Alver:2010dn}. The parameter $v_{n}$ is the magnitude of the $n^{\mathrm{th}}$-order complex flow coefficient $V_n$, defined as $V_{n} = v_{n}e^{in\Psi_n}$, and can be calculated according to 

\begin{equation}
v_{n} = \langle{\cos[n(\varphi - \Psi_n)]}\rangle,
\label{Eq:vn}
\end{equation}

where the brackets denote an average over all particles in all events. Since the symmetry planes are not accessible experimentally, the flow coefficients are estimated solely from the azimuthal angles of the particles emitted in the transverse plane. Measurements of different anisotropic flow coefficients at both RHIC \cite{Adams:2003am,Abelev:2007qg,Adler:2003kt,Adare:2006ti} and the LHC \cite{Abelev:2014pua,Adam:2016nfo,Acharya:2018zuq} have not only confirmed the production of a strongly coupled quark gluon plasma (sQGP) but they have also constrained the value of shear viscosity over entropy density ($\eta/s$) very close to the conjectured lower limit of $1/4\pi$ from AdS/CFT \cite{Kovtun:2004de}. In addition, both viscous hydrodynamical calculations and measurements \cite{Adam:2016izf} show that higher order flow coefficients and more importantly their transverse momentum dependence are more sensitive probes than lower order coefficients, i.e. \vtwo~and \vthree, of the initial spatial irregularity and its fluctuations.\\ % since higher order flow coefficients, originating from fluctuations of the initial state, probe smaller spatial scales.
%The non-vanishing higher order flow coefficients, $v_n (n>2)$, are believed to originate mainly from the fluctuations in the initial energy density profile of the colliding nucleons \cite{}. Thus, higher order flow coefficients are better probes to to understand the initial density profile of the 
The initial state spatial irregularity can be quantified with the standard (moment-based) anisotropy coefficients, $\epsilon_{n}$. Together with their corresponding initial symmetry planes, $\Phi_n$, the initial anisotropy coefficients can be calculated from the transverse positions of the participating nucleons as \cite{Alver:2010gr,Alver:2010dn} 

\begin{equation}
\epsilon_{n}e^{in\Phi_n} = \frac{\langle{r^{n}e^{in\phi}}\rangle}{\langle r^n\rangle}  (\rm{for}~n>1),
\label{Eq:epsilonn}
\end{equation}

where the brackets denote the average over the transverse position of all participating nucleons, $\phi$ is azimuthal angle and $r$ the polar distance. Model calculations show that $V_2$ and to a large extent, $V_3$ are  determined by their corresponding initial spatial anisotropy coefficients, $\epsilon_{2}$ and $\epsilon_{3}$, respectively \cite{Alver:2010gr}. It has been recently realised that $V_n~(n > 3)$ are not linearly correlated with their corresponding $\epsilon_{n}$ \cite{Bhalerao:2014xra}. In fact, a cumulant-based definition of initial anisotropic coefficient suggests additional terms in the definition of $\epsilon_{n}$ for higher order flow coefficients ($n>3$). As an example, the fourth order spatial anisotropy is given by \cite{Teaney:2013dta,Qian:2017ier} 
 
\begin{equation}
\epsilon_{4}'e^{i4\Phi'_4} = \epsilon_{4}e^{i4\Phi_4}  + \frac{3\langle{r^{2}}\rangle^{2}}{\langle r^4\rangle}\epsilon_{2}^{2}e^{i4\Phi_2}. 
\label{Eq:epsilonnprime}
\end{equation}

This dependence on lower order initial anisotropies gives rise to additional terms in the higher order flow coefficients. As a result, $V_n~(n > 3)$ obtain contributions not only from the linear response of the system to $\epsilon_{n}$, but also non-linear response proportional to the product of lower order initial spatial anisotropies \cite{Bhalerao:2014xra,Yan:2015jma}. 

It was shown in \cite{Acharya:2017zfg} that in a single event, $V_n$ with $n=4,5,6$ are decomposed to the linear ($V_{n}^{\mathrm{L}} $) and non-linear ($ V_{n}^{\mathrm{NL}}$) modes according to

\vspace{-0.55cm}
\begin{align}
V_{4} &= V_{4}^{\mathrm{L}} + V_{4}^{\mathrm{NL}} = V_{4}^{\mathrm{L}} + \chi_{4,22}(V_{2})^2, \nonumber \\
V_{5} &= V_{5}^{\mathrm{L}} + V_{5}^{\mathrm{NL}} = V_{5}^{\mathrm{L}} + \chi_{5,32}V_{3}V_{2}, \nonumber \\
V_{6} &= V_{6}^{\mathrm{L}} + V_{6}^{\mathrm{NL}} = V_{6}^{\mathrm{L}} + \chi_{6,222}(V_{2})^3 + \chi_{6,33}(V_{3})^2 + \chi_{6,42}V_{2}V_{4}^{\mathrm{L}},
\label{Eq:V4V5V6}
\end{align}
\vspace{-0.55cm}

where $\chi_{n,mk}$, known as non-linear flow mode coefficients, quantify the contributions of the non-linear modes to the total $V_{n}$ \cite{Acharya:2017zfg}.  
 
 Various measurements of total flow coefficients at the LHC \cite{Abelev:2014pua,Adam:2015eta,Adam:2016nfo,Acharya:2018zuq} and RHIC \cite{Adams:2003am,Abelev:2007qg,Adler:2003kt,Adare:2006ti} as a function of transverse momentum for different particle species have revealed that an interplay between radial flow and the total flow coefficients leads to a characteristic mass dependence in the low $p_{\rm{T}}$ region ($p_{\rm{T}} < 3~{\rm GeV}/c$). For higher values of $p_{\rm{T}}$ (up to $6~{\rm GeV}/c$) results for all total flow coefficients indicate a particle type grouping where baryons have a larger magnitude than the one of mesons. This feature was explained in a dynamical model where flow develops at the partonic level followed by quark coalescence into hadrons \cite{Voloshin:2002wa,Molnar:2003ff}. This model assumes that the invariant spectrum of produced particles is proportional to the product of the spectra of their constituents and in turn flow of produced particles is a sum of the flow of its constituents. Previous measurements of lower order total flow coefficients exhibit number of constituent quarks (NCQ) scaling at RHIC \cite{Adare:2012vq} and the LHC \cite{Abelev:2014pua,Adam:2016nfo} at an approximate level of $\pm20$\% for $p_{\rm{T}} > 3$ \GeV.

The measurements of non-linear flow modes in different collision geometries challenge hydrodynamic models to further constrain both the initial conditions of the collision system and its transport properties \cite{Acharya:2017zfg}. The \pT-dependent non-linear flow modes of identified particles are important observable for studying the characteristics of QGP. They not only put a stringent constraint on both the initial conditions of the collision system and its transport properties, i.e. $\eta/s$ and $\zeta/s$, but also allow to test the effect of late-stage interactions in the hadronic rescattering phase as well as the effect of particle production via coalescence mechanism to the development of the mass ordering and particle type grouping.

In this article, we report the first results of the $p_{\rm{T}}$-differential non-linear flow modes, $v_{4,22}$, $v_{5,32}$, $v_{6,33}$ and $v_{6,222}$ for \pion, \kaon, \proton, \Ks, \lambdas~and $\varphi$ measured in Pb--Pb collisions at the centre of mass energy per nucleon pair \sNN~with the ALICE detector \cite{Aamodt:2008zz} at the LHC. The detectors and the selection criteria used in  this analysis are described in Sec. \ref{Sec:ExpSetup} and \ref{Sec:EventTrackIdentification}, respectively. %The charged particles are identified using signals from both the Time Projection Chamber (TPC) and the Time Of Flight (TOF) detectors described in Section \ref{Sec:ExpSetup} with the procedures in Section \ref{SubSec:Identification} and neutral particles on a statistical basis using an invariant mass technique, also described in Section \ref{SubSec:Identification}. 
The results are obtained with a generic framework described in Section \ref{Sec:Analysis method} and in detail in \cite{Bilandzic:2013kga}. In this article, the identified hadron under study and the charged reference particles are obtained from different, non-overlapping pseudorapidity regions. The correlations not related to the common symmetry plane (non-flow), like those arising from jets, resonance decays and quantum statistics correlations, are suppressed by using multi-particle correlations as explained in Section \ref{Sec:Analysis method} and the residual effect is assigned as a systematic uncertainty, described in Section \ref{Sec:Systematics}. All coefficients for charged particles were measured separately for particles and anti-particles and were found to be compatible within statistical uncertainties. The reported measurements are therefore an average of the results for the opposite charges. The results are reported within the pseudorapidity range $|\eta|<0.8$ at different collision centralities between 0--60\% range of Pb--Pb collisions. 





