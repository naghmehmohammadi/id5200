\section{Systematic uncertainties}
\label{Sec:Systematics}

The systematic uncertainties were estimated by varying the selection criteria for all particle species as well as the topological reconstruction requirements for \Ks, \lambdas~and $\phi$. The contributions from different sources were extracted from the relative ratio of the \pT-differential $v_{n,mk}$ between the default selection criteria described in Sec.~\ref{Sec:EventTrackIdentification} and their variations summarised in this section. Sources with a statistically significant contribution (where significance is evaluated as recommended in \cite{Barlow:2002yb}) were added in quadrature to form the final value of the systematic uncertainties on the non-linear flow modes. 
An overview of the magnitude of the relative systematic uncertainties per particle species is given in Tab. \ref{SystematicsValues:PID} for \pion, \kaon~and \proton~and Tab. \ref{SystematicsValues:V0} for \Ks, \lambdas~and the $\phi$-meson. The systematic uncertainties are grouped into five categories, i.e.\,event selection, tracking, particle identification, topological cuts and non-flow contribution and are described below.

The effects of event selection criteria on the measurements were studied by:  (i) varying the primary vertex position along the beam axis ($z_{vtx}$) from a nominal $\pm$10 cm to $\pm$8 cm and $\pm$6 cm; (ii) changing the centrality estimator from the signal amplitudes in the V0 scintillator detectors to the number of clusters in the first or second layer of the SPD, (iii) analysing events recorded for different magnetic field polarities independently; (iv) not rejecting all events with tracks caused by pileup. 

Systematic uncertainties induced by the selection criteria imposed at the track level were investigated by: (i) changing the tracking from global mode, where combined track information from both TPC and ITS detectors are used, to what is referred to as hybrid mode. In the latter mode, track parameters from the TPC are used if the algorithm is unable to match the track reconstructed in the TPC with associated ITS clusters; (ii) increasing the number of TPC space points from 60 up to 90 and (iii) decreasing the value of the $\chi^{2}$ per TPC space point per degree of freedom from 4 to 3; (iv) varying the selection criteria on both the transverse and longitudinal components of the DCA to estimate the impact of secondary particles from a strict \pT-dependent cut to 0.15 cm and 2 cm to 0.2 cm, respectively.

Systematic uncertainties associated with the particle identification procedure were studied by varying the PID method from a \pT-dependent one described in Sec. \ref{SubSec:Track} to an even stricter version where the purity increases to higher than 95\% (\pion), 80\% (\kaon) and 80\% (\proton) across the entire \pT~range of study. The second approach relied on the Bayesian method with a probability of at least 80\% which gives an increase in purity to at least 97\% (\pion), 87\% (\kaon) and 90\% (\proton) across the entire \pT~range of study. To further check the effect of contamination the purity of these species was extrapolated to 100\%. 

The topological cuts were also varied to account for the \vo~and $\phi$-meson reconstruction. These selection criteria were varied by 
(i) changing the reconstruction method for  \vo~particles to an alternate technique that uses raw tracking information during the Kalman filtering stage (referred to as online V0 finder); (ii) varying the minimum radial distance from the primary vertex at which the \vo~can be produced from 5 cm to 10 cm; (iii) changing the minimum value of the cosine of pointing angle from 0.998 to 0.99; (iv) varying the minimum number of crossed TPC pad rows by the \vo~daughter tracks from 70 to 90; (v) changing the requirement on the minimum number of TPC space points that are used in the reconstruction of the \vo~daughter tracks form 70 to 90; (vi) requesting a minimum ratio of crossed to findable TPC clusters from 0.8 to 1.0; (vii) changing the minimum DCA of the \vo~daughter tracks to the primary vertex from 0.1 cm to 0.3 cm; (viii) changing the maximum DCA of the \vo~daughter tracks from 0.5 cm to 0.3 cm; (ix) requiring a minimum \pT~of the \vo~daughter tracks of 0.2 \GeV. 

In addition, the non-flow contribution was studied by (i) selecting like sign pairs of particles of interest and reference particles to decrease the effect from the decay of resonance particles; (ii) applying pseudorapidity gaps between the two subevents from $|\Delta\eta|>0.0$ to $|\Delta\eta|>0.4$.

Tables \ref{SystematicsValues:PID} and \ref{SystematicsValues:V0} summarise the maximum relative systematic uncertainties for each individual systematic source described above for all transverse momenta. The systematic uncertainties are expressed for each non-linear mode and particle species in a range to account for all centrality intervals in this article. 

\begin{table}[!ht]
\caption{List of the maximum relative systematic uncertainties of each individual source for $v_{\rm n,mk}$ of \pion, \kaon~and \proton. The uncertainties depend on the transverse momenta. Percentage ranges are given to account for all centrality intervals.}
\resizebox{\textwidth}{!}{\begin{tabular}{ |p{4.5cm} |l|c|c|c|c|c|c|c|c|c|c|c|c|}
\hline
\multicolumn{1}{| c |}{} & \multicolumn{3}{| c |}{ $v_{4,22}$ } & \multicolumn{3}{| c |}{ $v_{5,32}$} & \multicolumn{3}{| c |}{ $v_{6,33}$} & \multicolumn{3}{| c |}{ $v_{6,222}$} \\
\hline
Uncertainty source  & \pion &  \kaon & \proton &  \pion & \kaon & \proton &  \pion &  \kaon & \proton &  \pion &  \kaon & \proton \\ \hline  \hline
Primary $z_{vtx}$  & 0--2\% & 1--3\% & 0--3\% & 0--3\% & 1--3\% & 1--4\% & 3--5\% & 2--5\% & 3--5\% & 2--7\% & 2--7\% & 4--7\%\\
Centrality estimator  & 0--4\% & 1--4\% & 1--5\% & 0--4\% & 1--3\% & 2--4\% & 4--10\% & 4--10\% & 5--10\% & 3--10\% & 5--10\% & 4--10\%\\
Magnetic field polarity & 0--2\% & 0--3\% & 0--3\% & 0--4\% & 0--5\% & 0--5\% & 0--10\% & 0--10\% & 0--10\% & 0--10\% & 0--10\% & 0--10\% \\
Pileup rejection & 0--4\% & 0--3\% & 0--4\% & 0--5\% & 1--5\% & 0--5\% & 5--7\% & 5--10\% & 5--8\% & 4--10\% & 4--10\% & 2--10\%\\\hline
Tracking mode  & 1--4\% & 1--5\% & 1--4\% & 2--6\% & 3--5\% & 2--8\% & 0--8\% & 0--7\% & 3--8\% & 1--10\% & 4--10\% & 2--10\% \\
Number of \TPC~space points &  1--2\% & 0--2\% & 0--2\% & 0--3\% & 1--3\% & 1--3\% & 4--8\% & 3--8\% & 3--8\% & 2--8\% & 4--8\% & 4--8\% \\
$\chi^2$ per \TPC~space point & 0--2\% & 1--2\% & 1--3\% & 1--3\% & 1--3\% & 2--4\% & 3--5\% & 3--6\% & 3--6\% & 2--6\% & 4--7\% & 4--7\%\\
$\rm DCA_{xy}$ & 0--2\% & 0--2\% & 1--3\% & 0--3\% & 1--3\% & 1--3\% & 2--7\% & 2--8\% & 4--8\% & 2--8\% & 4--8\% & 3--8\%\\
$\rm DCA_{z}$ & 0--3\% & 0--2\% & 1--2\% & 1--2\% & 1--3\% & 2--3\% & 3--7\% & 3--7\% & 5--7\% & 2--7\% & 4--8\% & 2--8\%\\\hline
Particle identification & 1--5\% & 1--5\% & 1--3\% & 1--5\% & 2--5\% & 1--5\% & 5--10\% & 5--10\% & 6--12\% & 4--12\% & 6--15\% & 4--15\%\\\hline
POI vs. RFP charges & 0--2\% & 0--3\% & 2--3\% & 0--4\% & 0--4\% & 2--4\% & 0--4\% & 0--6\% & 0--6\% & 0\% & 0\% & 0\% \\ 
$\eta$ gap & 1--3\% & 1--4\% & 1--2\% & 1--4\% & 1--4\% & 1--5\% & 0--5\% & 0--5\% & 0--5\% & 0\% & 0\% & 0\%  \\\hline 
\end{tabular}}
\label{SystematicsValues:PID}
\end{table}

\begin{table}[!ht]
\centering
\caption{List of the maximum relative systematic uncertainties of each individual source for $v_{\rm n,mk}$ of \Ks, \lambdas~and $\phi$-meson. The uncertainties depend on the transverse momenta and centrality interval. Percentage ranges are given to account for all centrality intervals. "N/A" indicates that a certain check was not applicable to the given particle of interest. If a source was checked and proved to have a negligible effect, the field is marked as "--".}
\resizebox{0.75\textwidth}{!}{\begin{tabular}{ |p{6.7cm} |l|c|c|c|c|c|c|c|}
\hline
\multicolumn{1}{| c |}{} & \multicolumn{3}{| c |}{ $v_{4,22}$ } & \multicolumn{2}{| c |}{ $v_{5,32}$} & \multicolumn{2}{| c |}{ $v_{6,33}$} \\
\hline
Uncertainty source  & \Ks &  \lambdas & $\phi$ &  \Ks &  \lambdas & \Ks &  \lambdas   \\ \hline  \hline
Primary $z_{vtx}$  & 0\% & 0-2\% & 1\% & 0\% & 0--3\% & 0\% & 1--3\%\\ \hline
Tracking mode  & - & - & 2\% & - & - & - & - \\
Number of \TPC~space points & 0--3\% & 1--2\% & 2\% & 0\% & 2\% & 0\% & 2\%  \\ \hline
Particle identification & - & - & 4--6\% & - & - & - & -\\\hline
Reconstruction method (\vo~finder) & 3--5\% & 2--3\% & N/A & 5\% & 1\% & 5\% & 1\%   \\
Decay radius & 3--5\% & 1--3\% & N/A & 5--6\% & 0--2\% & 5\% & 2\%\\
Ratio of crossed to findable TPC clusters & 0--2\% & 0--3\% & N/A & 0\%  & 1--2\% & 0\% & 3\%  \\
DCA decay products to primary vertex & 2--5\% & 2--4\% & N/A & 4--5\% & 2--3\% & 5\% & 2--3\%  \\
DCA between decay products & 0--3\% & 1--2\% & N/A & 0--4\% & 0--4\% & 0\% & 0--4\% \\
Pointing angle $\cos(\theta_{\rm p})$ & 3--4\% & 0--2\% & N/A & 3--4\% & 0--3\% & 3\% & 1\%  \\
Minimum \pT~of daughter tracks & 1--3\% & 0--1\% & N/A & 2--3\% & 2--3\% & 0\% & 0--3\%  \\ \hline
\end{tabular}}
\label{SystematicsValues:V0}
\end{table}
